\documentclass[10pt,letter]{article}

\author{Dustin Lang}

\begin{document}
\maketitle

\section{Introduction}

This document describes the iterative design of a deblender through
experiments.  It is presented as a sequence of refinements guided by
looking at the results on real data.

\section{Starting ansatz}

The initial ansatz, due to Robert Lupton \cite{rhldeblend} and
implemented in the SDSS \emph{Photo} software, is that astronomical
sources have two-fold rotational symmetry and a peak in the center.
This is a fiendishly clever observation that will take us a long way.
We will begin by describing the SDSS deblending algorithm built around
this ansatz, sweeping several details under the rug until later.

The SDSS deblender takes as input a ``footprint'' and a set of
``peaks''.  A footprint is a connected set of pixels that are
significantly above a detection threshold, grown by a margin of about
the size of the PSF.  ``Peaks'' are just what you expect:
(PSF-convolved) pixels significantly higher than their neighbors.

For each peak, the SDSS deblender builds a ``template'' image by
applying the symmetry ansatz.  Starting from the peak pixel, at
position $(p_x,p_y)$, the template at positions $(p_x + dx, p_y +
dy)$, $(p_x - dx, p_y - dy)$ contains the \emph{minimum} of those two
pixel values.  The difference between the minimum and the value at the
higher pixel is presumed to be due to other blended sources.

Having built a template for each peak, the SDSS deblender computes the
least-squares \emph{weight} for each template to best reproduce the
observed image.


\end{document}

