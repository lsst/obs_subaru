
\subsection{Introduction}

  The job of the Make Starlist module is to turn the set of all
postage stamps created by the on-line system into a list of 
stars with measured properties.  The Find PSF module will 
extract an appropriate PSF from this set of measured stars.

  The flow in this module is:
\begin{itemize}
\item {\it For each color:}
\begin{itemize}
\item {\it For each frame:}
\begin{itemize}
\item {\it For each postage stamp:}
\begin{itemize}
\item {\it Read in a postage stamp}
\item {\it Flatfield the postage stamp}
\item {\it Create a STAR1 structure which includes the postage stamp}
\item {\it Calculate simple position, shape and photometry information}
\item {\it Add STAR1 to list}
\end{itemize}
\item {\it End of stamp loop}
\end{itemize}
\item {\it End of frame loop}
\end{itemize}
\item {\it End of color loop}
\end{itemize}

The measured parameters are: 
intensity-weighted first moments are used to calculate the position, 
intensity-weighted second moments used for the shape,
and sky-subtracted counts in a circular aperture used for photometry.
The sky is determined via the {\tt shRegSkyFind} function,
which fits a gaussian to the histogram of all pixel values
inside the postage stamp.
