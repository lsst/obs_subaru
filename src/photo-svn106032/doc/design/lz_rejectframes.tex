
\subsection{Introduction}

There are some REALLY bright objects in the northern hemisphere that Jim
Gunn thinks will produce bizarre and unpredictable results in the CCD.
This module is intended to catch those affected frames before one has to
run an object finder of any sort on it.

\subsection{Input}

\subsubsection{List of very, very bright objects}
This must include the position in great circle coordinates and the
exclusion radius of the object.  It probably includes other information
like what the object is.

\subsubsection{Row of start or center of red frame}
\subsubsection{Transformation parameters from red to great circle coordinates}

\subsection{Output}

\subsubsection{Galactic coordinates of the frame}

\subsubsection{Flag}
The flag is true if the frame is within the exclusion area of the bright
object, false otherwise.

\subsection{Modules}

This is easily broken into two modules: one which calculates the galactic
coordinates of the center of the frame, and one which checks the galactic
galactic coordinates against the objects in the bright object list.

\subsection{Algorithms}

\subsection{Quality, Debugging, Resources}

