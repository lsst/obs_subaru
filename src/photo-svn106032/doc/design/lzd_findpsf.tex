
\def \fpsf{{\bf Find PSF\ }}
\def \fbs{{\bf Find Bright Stars\ }}

\subsection{Introduction}

  This module takes the list of stars created by the Make Starlist
module, examines them all, throws out the bad ones, and uses
the rest to create a PSF representative of the stars in the list.
We currently use stars from three consecutive frames to determine
the PSF for the middle frame (we cannot be sure that each frame
will have enough stars to guarantee a fair sample).
The Level Zero PSF is a sum of two gaussians, each allowed to have
different widths in the row and column directions.
We do not allow the PSF to vary as a function of position on the chip.

  Note that the Level Zero pipeline does {\it not} attempt to 
model the extended wings of bright stars.

\subsection{Algorithms}

  This module makes no use of {\it a priori} information about the
expected PSF; it uses the properties of the stars which are given 
to it.  Therefore, a pathological set of postage stamps will 
result in a pathological PSF.

  For each star, the code first attempts to find a double
gaussian that best fits it.  The initial guesses use the
second moment of the star: the ``inner'' gaussian has FWHM 
roughly half that of a single gaussian fitted to the star, 
and the ``outer'' gaussian roughly twice.  The amplitude of
``inner'' gaussian is given an initial value ten times larger
than the ``outer'' one.  A non-linear least-squares fit is
then allowed to find the best values for the parameters of the
two gaussians.  
{\it (note in proof: these initial guesses are currently
hard-wired into the code; we ought to make them tunable parameters). }

  Once each star has had a fit made to it individually, the
entire list of stars is passed to a second routine that attempts
to find a ``mean'' value for each of the five DGPSF parameters
(inner width in row and col, outer width in row and col, and ratio
of amplitudes).  We use a simple iterative method in which 
outliers are discarded after each iteration.

   The final PSF is returned and set in one of the CALIB1BYFRAME
structures.

   We find that the PSF determined in this fashion is 
consistent and robust, but always wider than the input PSF 
used in creating the simulations.  We suspect that this is due
in part to the extended power-law wings in the simulations
(which we do not model), and in part due to using too large 
a fitting radius.

