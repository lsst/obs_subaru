\def \bom{{\bf Find Bright Objects\ }}  
 
\subsection {Introduction}
 
The goal of the \bom module is to identify all objects on a given
frame (one color) above a certain threshold. Examples of bright
objects are bright stars, ghosts, bleed trails and satellite
tracks. Identified objects are grown to a second threshold (defined in
terms of the standard deviation of the background sky). After
completion of the identification of the bright objects bleed trails
from the saturated objects are identified and masked.
 
\subsection {InitModule}
 
The initialization module sets the parameters required by the main
module for identifying and growing bright objects. It also sets the
parameters used for identifying bleed trails.
 
The {\bf InitModule} reads parameters from the ``parameters file''
into memory which is private to the module.  
 
\begin{itemize}
 
  \item {\bf fbo\_peakval} : the threshold (in terms of the standard
deviation of the sky) above which an object must have a pixel value for
it to be described as bright.
 
  \item {\bf fbo\_threshold} : the threshold per pixel (in terms of the
standard deviation of the sky) to which an object is grown after detection.
 
  \item {\bf fbo\_bleed} : the level above the sky (in sky sigma) at
to which a bleed trail from a bright object is detected.
 
\end{itemize}
 
\subsection{Input}
 
The \bom module requires the following inputs,
 
\begin{itemize}
 
\item A flatfielded, bias corrected image frame (currently 2048 by
1460 pixels).
 
\item A noise image frame giving the Poisson noise for each pixel in
the corrected frame.
 
\item An estimate of the sky background across the frame. 
 
\item An estimate of the standard deviation of the sky background
across the frame.  
 
\item A list of OBJECT1 structures (NULL on input).
 
\item A FRSTAT structure describing the statistics of the frame.
 
\end{itemize}
 
\subsection{Output}
 
On completion the \bom module outputs for each color frame,
 
\begin{itemize}
 
\item An image frame with the mask pixels identifying the positions of
the bright objects and saturation trails.
 
\item A noise frame with all saturated pixels and bleed trails
identified by setting the noise value to its maximum.
 
\item A list of OBJECT1 structures comprising the bright objects
identified across the frame.
 
\item A FRSTAT structure identifying the number of bright objects and
bad pixels found.
 
\end{itemize}
 
\subsection{Modules and Algorithms}
 
The \bom module comprises two parts, identifying the peaks of the
bright objects and growing the objects to a second threshold and
identifying bleed trails across a frame.
 
Each image frame is scanned. All pixels with values above the
{\bf fbo\_peakval} are marked as the position of a bright object.
Each of the marked pixels is passed to the {\bf build\_bright}
module. This uses a connected pixel algorithm to grow the bright
objects to second threshold ({\bf fbo\_threshold}) above the sky. The
current default setting is 1 $\sigma$ per pixel above the sky. The {\bf
build\_bright} module returns an OBJECT1 structure on completion.
 
Each pixel that has been identified as belonging to a bright object is
flagged in the image and OBJECT1 masks (as MASK\_BRIGHTOBJECT and
MASK\_OBJECT).
 
Bleed trails are identified by searching the input frame for saturated
pixels (the saturation level is defined globally). When found the algorithm
sets a variable thesehold and begins to step down a column. If the pixel
values decrease then the threshold decreases. When the threshold reaches
a level {\bf fbo\_bleed} above the sky the process stops and is repeated    
in the opposite direction. Identified bleed trails are flagged as having 
infinite noise (in the noise mask) and as MASK\_NOTCHECKED in the region mask.
 
\subsection{Quality, Debugging, Resources}
 
Quality Assurance is a very difficult task for this module, because
the number, relative frequency and severity of bright stuff varies
greatly from one frame to the next. The module returns in the FRSTAT
structure the number of bright objects and bad pixels found. This can
be compared with the expected number of bright sources and cosmic ray
event rates in order to access whether 
 
For debugging and QA, it is possible to
display an image and blink on/off all those pixels
belonging to identified Bright Objects, saturated pixels and bleed
trails.
