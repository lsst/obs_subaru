 
\subsection{Introduction}
 
This module sets up the frame for faint object finding.  It must 
mask and/or subtract the stars, and extract postage stamps.
{\bf Note that the defects introduced by bright objects will
remain uncertain until the optics of the telescope system are well
understood (i.e.\ the test year). Fitting of PSF's and subtraction of
bright stars (including the extended wings of the PSF for very bright
stars) from the corrected frames will need
careful regression testing.
It is likely, therefore, that this module and {\bf Bright Objects}
will require substantial
man-power and computational resources during the
Level One and test year phases.}
 
\subsection {InitModule}
 
\subsubsection {Description}
 
The initialization module sets the parameters required by the main
module for subtracting model images from a frame.
 
\subsubsection {Input}
 
The {\bf InitModule} reads parameters from the ``parameters file''
into memory which is private to the module.  
 
\begin{itemize}
 
  \item {\bf cf\_noisecut} : the noise level per pixel (in terms of
the standard deviation of the sky) at which we cannot usefully
subtract a model.
 
  \item {\bf cf\_subsig} : the threshold per pixel (in terms of the
standard deviation of the sky) out to which an object is subtracted.
 
\end{itemize}
 
\subsubsection {Output}
 
  There is no output under normal operation.
 
\subsubsection {Modules and Algorithms}
 
  {\bf InitModule} simply reads parameter values into memory.
 
  Parameters are read in by calling the ftcl GetPar routines.
 
\subsubsection {Quality, Debugging, Resources}
 
  The {\bf InitModule} calls the standard DERVISH error routines if a
parameter cannot be read.
 
  In ``debug mode,'' the function prints out the value of
each parameter as it succeeds in reading it from the parameters file.
 
  This routine consumes negligible resources.
 
\subsubsection {Test Data Required}
 
  There is no test data required.
 
\subsubsection {Regression Testing}
 
  The ``debug mode'' output should suffice for regression testing.
 
\subsection{Input}
 
\begin{itemize}
 
\item An array of corrected frames of all colors.
 
\item An array of noise frames for all colors.
 
\item A list of classified OBJC structures.
 
\end{itemize}
 
\subsection{Output}
 
\begin{itemize}
 
\item An array of corrected frames with bright objects subtracted or masked.
 
\item An array of noise frames with regions of high noise 
({\bf cf\_noisecut}) set to infinite.
 
\item A list of OBJC structures. Objects subtracted from a frame have their
atlas images extracted before subtraction.
 
\end{itemize}
 
\subsection{Modules and Algorithms}
 
Objects classified as stars by bright objects classify are subtracted
from a frame. All other objects are masked on a frame. The modules 
called are :
 
\begin{itemize}
 
\item subObject : Subtracts the fitted PSF (for objects classified as a 
star). Objects are subtracted to {\bf cf\_sigmacut} times the standard 
deviation on the sky. For regions with noise values (determined from
the array of noise regions) greater than {\bf cf\_noisecut} the array
is masked as MASK\_NOTCHECKED to avoid being identified by the faint
object finder.
 
\item maskObject : All objects that are not classified as an isolated star
are masked. Each pixel identified as an object (MASK\_BRIGHTOBJECT) is
set to (MASK\_NOTCHECKED) to avoid detection by the faint object finder.
 
\end{itemize}
 
 
\subsection{Quality, Debugging, Resources}
 
This module requires substantial quality, debugging and regression tests.
 
