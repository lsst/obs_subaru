
\def \fbs{{\bf Find Bright Stars\ }}

\medskip
\centerline {\it authors: Andy Connolly and Michael Richmond}

\subsection{Introduction}

The \fbs module identifies, measures and either 
subtracts or masks all
bright stars on a frame, including mildly saturated and unsaturated ones.
The main goal is to identify isolated, unsaturated stars with which
to perform both astrometric and, more importantly, photometric calibration.
In the Level Zero design there is a substantial overlap between the goals
of the \fbs module and {\bf Bright Objects}. It is
envisaged that these two modules may be combined at Level One. For
clarity of programming, however, the modules will remain separate for the
current version of the pipeline, with \fbs focusing
on working correctly and accurately with ``good'' bright stars
in preference to dealing with those rare, heavily saturated objects.

{\bf Note that the defects introduced by bright objects will
remain uncertain until the optics of the telescope system are well
understood (i.e.\ the test year). Fitting of PSF's and subtraction of 
bright stars (including the extended wings of the PSF for very bright
stars) from the corrected frames will need 
careful regression testing.
It is likely, therefore, that this module and {\bf Bright Objects}
will require substantial 
man-power and computational resources during the Level Zero
and Level One phases.}

\subsection {Module}

\subsubsection {Description}

  The main \fbs module has four main goals, plus one that may or may
not be required in the Level Zero Pipeline:

\begin{itemize}
  \item locating and measuring all the ``bright'' stars in the main 
        photometric images (where ``bright'' is defined by the
        input parameters to the module)
  \item creating {\it STAR1} structures for each star, filling in
        the appropriate information and adding the structure to 
        a linked list 
  \item marking the {\it MASK} values for each pixel which belongs
        to a star
  \item attempting to match each star with the Known Objects list
  \item (optional) writing postage stamps around each star out to
        the framework and then subtracting a model fit to each
        star individually.  This task may require extra work to deal
        with saturated stars, and with the outer wings of very bright
        stars.  It must also set {\it MASK} values appropriately in
        the star-subtracted regions.
        
\end{itemize}

  The final item may be sub-divided into several pieces, some of which
(dealing with the extended wings of very bright stars) may not be
implemented in Level Zero.

The central question regarding the \fbs module is the number and
magnitude of stars expected per frame. If we assume the Gilmore and Wyse
Galaxy model, the number of stars per 2048$\times$1500 (0.038 deg$^2$) CCD
frame can be calculated as a function of Galactic latitude. 
The current frame size of the CCD presents no problems for 
the low latitude regions ($<45^\circ$) 
with $\sim 11$ per frame (assuming we 
measure bright stars between 14th and 16th --- extending to 17th mag 
gives $\sim 22$).
At the NGP this drops to $\sim$ 6 and 11 respectively. The situation is worse
in the {\it u'} band (current models do not extend into the 
{\it u'} but numbers can be
estimated for a solar metallicity star with U-V = 0.8), 
where the numbers at the
NGP are $\sim$ 3 and 6. The problems this introduces in the image alignment
are addressed later.

\subsubsection{Input}

  \fbs requires

\begin{itemize}
  \item a corrected frame (including header information such as the
        time of the exposure)
  \item a catalog list of known bright objects in the frame (created by the
        {\bf Research Known Objects} module), or those close enough
        to produce visible effects (diffraction spikes, bright wings)
  \item the {\it PSF} structure (determined by the {\bf InitModule}
        section)
  \item a small set of parameters defining the limits of ``bright''
        stars.  For example, the minimum peak value a star may have,
        the greatest deviation from the PSF allowed, the distance
        by which it must be separated from all other objects, etc.
  \item a set of parameters describing the data value or number of
        sigma above the sky above which pixels are regarded as BAD (even after
        subtraction of the PSF) due to the Poisson noise being too great to
        efficiently detect underlying data.
\end{itemize}

\subsubsection {Output}

  The output of the main \fbs module consists of

\begin{itemize}
  \item a linked list of {\it STAR1} structures, with position and
        magnitude fields filled appropriately and a pointer to bright
        stars identified by Research Known Objects. 
  \item a modified {\it MASK} structure for the current frame:
        the modifications are of two types
  \begin{itemize}
    \item the module sets a bit for each pixel belonging to a star
          which indicates ``this pixel belongs to some object''
    \item it also sets a pair of bits to indicate the state of
          pixels belonging to stars as BAD if they are saturated
          or have values so high that the shot noise precludes
          any further searching
  \end{itemize}
\end{itemize}

This list includes all ``bright'' unsaturated stars
and possibly including slightly saturated stars whose 
wings can be fit to the required precision.

  If the module also subtracts a fitted model of each star from 
the image, it must produce in addition

\begin{itemize}
  \item a postage stamp around each star, for the framework to
        store away somewhere
  \item a modified {\it REGION} structure for the current frame,
        in which the model has been subtracted
  \item a further modified {\it MASK} structure, in which
        all pixels from which a model has been subtracted are
        marked as either FIXED or BAD, depending on the 
        amount of noise contributed by the star at that point
\end{itemize}

\subsubsection {Modules}

  This module sub-divides nicely into several smaller tasks:

\begin{itemize}
  \item identifying possible bright-star candidates in a frame (and 
        possibly cutting out smaller regions around each for further
        operations)
  \item calculating some parameters for the candidate (aperture
        photometry needs to be coordinated with the Monitor Telescope Pipeline)
  \item deciding if it is a star or not
  \item if it {\it is} a star, then the module must
  \begin{itemize}
    \item modify the {\it MASK} values for pixels belonging to the star
    \item create a {\it STAR1} structure and fill with the
          appropriate information
    \item if we decide to subtract bright stars,
    \begin{itemize}
      \item write a postage stamp out to the pipeline framework
      \item subtract a fitted PSF away from the frame (recording the
            parameters of the PSF model used in the subtraction)
      \item modify the frame {\it MASK} values accordingly
    \end{itemize}
  \end{itemize}
\end{itemize}

\subsubsection {Algorithms}

First, {\it finding the bright stars}.
The routine we require for finding bright objects 
needs to be simple and fast, to maximize the
time available for object finding. The processing might be as follows:

\begin{verbatim}
     Scan array for pixels 100 sigma above the sky noise
                        |
        Tag these pixels in a temporary mask
                        |
     For each tagged pixel find all connected pixels
     down to 0.5 sigma of the sky and write to a structure
                        |
  Identify saturated pixels and calculate the moments Wxx Wxy Wyy
               of the connected pixels
                        |
  For star-like objects, fit the model PSF to the object to 
    determine whether it's really a star, and also to
    measure its position and brightness most accurately
\end{verbatim}

where the values ``100'' and ``0.5'' are examples of the
parameters given as input to this module (see {\bf Input} section).


Second, {\it fitting candidates to a stellar PSF}.
We certainly will need to use a more accurate representation 
for the PSF than a simple Gaussian.  It is unclear at the present
time whether a double Gaussian, a Gaussian plus a table of
residuals (as in DAOPHOT), or some other form will be required.  Note that 
the Level Zero {\it PSF} structure is relatively flexible, so
that we can make extensive tests without having to change
the pipeline framework.

Third, {\it classifying candidates as bright stars or non-stellar objects}.
Isolated stars are derived from the list of unsaturated stars by calculating
the ellipticity of each star. Stars with ellipticity $<0.25$ 
(again, this is an example of the ``parameters'' mentioned above)
are defined to
be isolated. The PSF is fitted for each star and 
the $\chi^2$ derived for each fit. 
Candidate stars are then accepted or rejected depending on
the quality of fit. For those stars identified by Research Known
Objects a pointer or id is stored to allow comparison photometry and 
astrometry.

Fourth, {\it subtracting bright stars}. The PSF determined by {\bf
Find PSF}
must be scaled and subtracted from each star on the corrected frame
(including partially blended stars -- though this will be left until
Level One). The model of the PSF and scaling must be recorded 
by outputting it to the frame work as header information for each
stellar postage stamp and/or as an additional structure in STAR1. 
\subsubsection {Additional problem areas}

{\it Saturated Stars:}
Saturation of stars (m$<$14) introduces two problems, fitting of the PSF to
determine the extent of a star and ghosting due to internal reflections.

\begin{itemize}
  \item Saturation : The number of saturated pixels in a star will be
kept in a temporary structure. The log area of these pixels gives
a rough estimate of the magnitude of the star and, therefore, the
probability that diffraction spikes and ghosts will require removing
(this can be tuned 
during the test year). Fitting the PSF to the unsaturated region of
the star (e.g.\ from 5 $\sigma$ above the sky to 90\% of the
saturation level) the extent of the star can be determined. The scaled
PSF can then be masked or subtracted from the corrected frame. For
heavily saturated stars we require a model of the extended wings of
the PSF (this may not be implemented in the Level Zero Pipeline).

Fitting of a PSF that extends above the saturation level is simplest if
performed in floating point,
so we may need to make a temporary copy of the {\it REGION} in question
with floating-point, rather than integer, values.
As we expect $<1$  such star every 10 
CCD frames (at 45$^\circ$)
this will not add significantly to the memory or overhead in the processing.

For saturated stars off the frame the diffraction spikes may still extend over
more than one frame. We will need to be able to quantify what
magnitude range this will occur for and, from the bright star catalog,
identify which frames will be affected. There will be a color
dependence in this (and all) defects where saturation/diffraction
spikes may occur in only one passband. As it is unlikely that we will
have multi--color information for all bright sources in the survey
(i.e.\ the HST Guide Star Catalog contains only b and v passbands) 
it may be possible to use the Known Objects lists 
only to warn of potential defects on a frame. 


  \item Ghosting: In direct imaging ghosts arise from bright stars producing
internal reflections in the camera. In principle the ghosts can be mapped
during the test year as a function of position on the chip. In drift scanning
it is not clear in what form ghosting will arise. It may only be possible to
flag the presence of stars capable of inducing ghosting so the module
calculating the sky distribution can search for gradients.

The {\bf Research Known Objects} module may be able to calculate 
ahead of time areas in which ghosts from bright stars are likely 
to occur, which will make the task of this module lighter.  
However,
it is unlikely that we will know enough about the optics of the 
telescope to do this in Level Zero.


  \item Image Alignment:
The number of stars per frame will vary as a function of galactic latitude,
from a few to a few tens. All stars used in defining the PSF
(isolated and unsaturated) are 
defined in the output linked list of
stars. The
centroid of these stars is accurately determined ($\sim 0.05$ arcsec from the
Gaussian fitting of the PSF). A problem occurs for high Galactic Latitude
(particularly in the {\it u'} band). 
To a magnitude limit of m$<$17, frames will exist
with less than the required 6 stars needed to derive a solution for the image
transformation. Decreasing the peak level at which the \fbs module
identifies stars is feasible but time consuming,
and in effect it is simply repeating the work of the 
{\bf Find Objects} module. 
Therefore, if there are less than 6 stars per frame an
error will be flagged 
and noted via standard pipeline error-logging functions.
The {\bf Offset Frames} module must be capable of either deriving
transformations dependent on the number of bright stars or using the
input from {\bf Find Objects}.

\end{itemize}

\subsubsection{Quality, Debugging and Resources}

The module should be able to print out a verbose description of
its actions (finding candidates, fitting them, discarding some)
as it executes, if desired.
Additionally, it should print a summary of the total number of candidates,
and number of actual stars, in the current frame.
This simple summary should suffice for Quality Assurance during runtime.

In debugging or verbose mode the frame must be displayable in 
real time, with
stars identified on screen and masked pixels marked (including saturated
pixels). The PSF profile needs to be displayed with the residuals of each star
(as a radial profile) together with the input PSF. Regions masked 
for saturated stars and and as tails of diffraction spikes need outlining.

During debugging in Level Zero (and probably through the test year) 
the \fbs module will retain the option
to derive a PSF from identified stars on a given frame. The number of
bright stars 
output by the online telescope system will be limited (to $\sim 10$
per frame). Typically at intermediate latitudes ($\sim 45^\circ$) more
than double this number will be available from which to derive the
PSF. A comparison between the time and spatial dependence of the PSF
for individual frames will provide a powerful tool for quality assurance. 

The \fbs module requires a mask array (2045$\times$1500) to
identify peaks and workspace to build connected pixels. Assuming stars will
occupy 30$\times$30 boxes and typically 10--20 per frame will be identified
leads to a requirement of 3 KiloBytes 
per object or 60 KiloBytes per frame. Comparisons of the CPU
required to calculate boxes as opposed to deriving linked lists of all
connected pixels is in progress.

Fitting a PSF to each good stellar candidate will require an
amount of CPU time large compared to other tasks, but, considering
the very small numbers of candidates on most frames (fewer than 20),
negligible on the scale of the entire pipeline.  

{\it If} we decide to subtract a fitted model for each star
from the frame, then the \fbs module must send a large amount of
data to the framework for each frame.  It must cut out a postage
stamp around each star before subtracting it, and give the stamp
to the framework for storage in the database.  Assuming that
all the bright stars fit within the boxsize mentioned above,
this comes to only about 60 KiloBytes per frame.  However,
if we try to model and subtract the extended wings of very
bright stars, then we may need to extract much larger
regions;  fortunately, such very bright stars will occur
very infrequently.

\subsubsection {Test Data Required}

The most important set of test data is that which allows us to
verify that the candidate-selection and PSF-fitting sections of
the module are working correctly.  Thus, we need

\begin{itemize}
  \item a set of frames with a number of bright stellar objects {\it and}
  \item a catalog of the precise positions and magnitudes of those stars
\end{itemize}

This data set must be produced by simulations.

A secondary consideration is a set of images with bright non-stellar
objects as well as bright stellar objects, so that we can verify that
the discrimination techniques are satisfactory.  

Finally, we ought to have a set of images with ``problematic'' 
objects --- very strongly saturated stars, which show bleed trails
and diffraction spikes, and ghost images --- so that we can test 
the code that deals with them.  

\subsubsection {Regression Testing}

In order to check that a new version of \fbs accurately reproduces
the output of some older, presumably correct version, 
we will need some kind of tool for comparing lists of numerical
data in a ``fuzzy'' fashion; this tool will be required for
Regression Testing of many modules.  

It would be good to have a standard set of test
images, or even a single one, with a variety of bright stars
and other objects mixed together.  We could then run the
module on that image(s) alone to check that it produces the
correct positions and magnitudes.  Note that we would also 
need a standard {\it PSF} structure and all the other inputs
for that particular frame to run the tests.

In the worst case, we may be able to use the summary of stars
found on a frame as a very quick and dirty measure of its
accuracy.

\subsection {FiniModule}

\subsubsection {Description}

The only task of the {\bf FiniModule} for \fbs is to de-allocate
any ``private'' memory that has been been used during the entire
course of pipeline execution.

\subsubsection {Input}

  There is no input (aside from the knowledge of the memory
which was allocated).

\subsubsection {Output}

  Under normal operation, there will be no output if all goes well.
Errors, of course, will be flagged by the usual pipeline
error-handling routines, which will print some kind of
error message.

  See the note on ``Debugging'' below, however.

\subsubsection {Modules}

  There's only one task to perform.

\subsubsection {Algorithms}

  Nothing fancy needed.

\subsubsection {Quality, Debugging, Resources}

  For debugging purposes, it might be useful for the {\it FiniModule}
to produce a summary of the complete activity of \fbs over the
entire run.  This would require the \fbs module to keep track
of various statistics in private memory and update them constantly
as it worked.

  No significant memory or CPU will be needed.

\subsubsection {Regression Testing}

  This {\bf FiniModule} is so minimal that no real Regression
Tests are required.  One could use a simple ASCII comparison
program to compare the summary statistics list produced
in ``debugging mode'' with similar lists from previous runs.

% end of Find Bright Stars section

