\subsection{Pipelines}

	The photometric pipeline has been divided into two sub-pipelines:
a postage stamp pipeline and a frame pipeline.  The postage stamp pipeline
will act on the postage stamps from the photometric camera, the quartile
data, and the output of the astrometric and monitor telescope pipelines.
it is responsible for rough photometric calibration,
determining the point spread function(s), and creating the flatfields.
After this sub-pipeline is run, the results will be evaluated to determine
whether they are reasonable.  If they are (the zero points haven't changed
tremendously, for example), then we start the frame pipeline process.
The frame pipeline takes the output of the postage stamp pipeline and the
five color images from a run and produces the catalog of found objects,
among other things.  Both of these sub-pipelines operate on one scan line
from one run at a time.  Each of these sub-pipelines should have one 
preprocessor, one processor, and one postprocessor which run in series.
At the present time, the pre- and post- processors are not implemented.

	The postage stamp and frames pipelines most naturally operate 
on units of complete scan lines (one continuous data stream (run) from one 
column of CCD's).  However, the data from the telescope will be written
in sets consisting of an entire night of data.  These data will be
archived into the survey data base.  The preprocessor
will query the data base and generate all of the (ascii) input files
for the processor, including the list of images in the complete
scan line and the photometric parameters to be applied to that
data set.

	The processor takes the output of the preprocessor plus the
raw data from a night's observing.  In the case of the postage stamp
pipeline, it generates rough astrometric calibrations, photometric
calibrations, flatfields, and the point spread function.  The frame
pipeline processor generates corrected image frames,
object lists, object parameters, and quality assurance data.  The processor 
should be capable of reading the raw data from tape or disk.  The frame
processor, since it operates on each frame independently, should be 
capable of starting and finishing in the middle of a run.

	The level zero processor outputs lists of objects and other output
as flat files.  The postprocessor will take this data and load it into the
database (after inspection of the quality assurance data from the processor).
In the case of the frames pipeline, the flatfiles are
generated on a frame by frame basis, which will produce object
duplication in the overlap regions.  It is possible that the postprocessor
for the photometric pipeline could merge the frame to frame overlap objects.

\subsection{Data flow in the Postage Stamp Pipeline Processor}

	The main program of the processor is a TCL script which accomplishes
the tasks in the accompanying figure.  The main scientific tasks are C
subroutines that are interfaced to TCL.

%\epsfxsize=0.98\textwidth
%\vspace{-1.5cm}
%\epsfbox{ps_pipeline.eps}
\epsfigure{ps_pipeline.eps}{}{}

\subsection{Input to the Postage Stamp Processor}

The level zero photometric pipeline reads its input as a set of FITS and 
ASCII files.  The ASCII files all have the form of one parameter and its
value per line of the file.  This format is used for plans and parameters.
The FITS files are either FITS images, or tables.  Currently, all of the
tables are binary tables.

\subsubsection{Data Processing Plan}

	The data processing plan identifies the other files to be used in
the processing, a list of data tapes required, and a resource plan (tape
devices, disk directories, etc.) as well as the mode of operation (this is
where we can specify debugging, use only two colors, or
normal mode, for example).

\subsubsection{Software Parameters}

The science modules are allowed to access files of parameters,
templates, etc.  These files can be private to the module in the sense
that no other piece of the pipeline depends on them, and the module
writer is responsible for opening and closing the file in the
init/finiModule code.  The name of this
file is listed in the Data Processing Plan.
This file is an ASCII file with the variable and its value on one line.

\subsubsection{Output from the MT pipeline}

From the MT pipeline, we need a FITS file containing the secondary
photometric standard stars, including their position in the sky, in some
coordinates.  We also need a FITS file containing the atmospheric
extinction in each band and the time at which that atmospheric extinction
was determined.

\subsubsection{Output from the astrometric pipeline}

We require a FITS table giving the linear transformation from row and column to
the coordinates in the MT secondary standards catalog for each color.

\subsubsection{Input from database}

We need information about the CCDs used, and all of the other information
that is typically found in an image header.  We have made a decision not
to use the actual image headers since they are notoriously incorrect and
since any test data will not necessarily contain the right keywords.
The CCD parameters file is currently a flat ASCII file, but could easily
be converted to a FITS file.  The other file is a binary FITS table of
FRAMEINFO structs which contain the frame number in the run, the airmass,
and the UT observation time of the image.

We also need the most recent photometric calibration to use in case we
cannot match any of the secondary calibration stars.

\subsubsection{Input from the DA system} 

We currently read in postage stamp and quartile disk files, one of each
from each frame in the run.  Both are binary FITS tables.  We also read
in two-dimensional fits images (one for each color) that are bias runs
on the telescope.  The postage stamp pipeline will reduce these into
bias vectors to be used with the imaging data.

\subsection{Scientific output from the Postage Stamp Pipeline Processor}

\subsubsection{Flat field vectors}

One flatfield vector is output for each frame in the run (where each frame
is 1354 rows long) and output as a separate FITS image.

\subsubsection{Bias vectors}

One bias vector is produced for each CCD, and this bias vector is
used for the whole run.

\subsubsection{Calibration parameters}

One set of calibration parameters per frame is output.  The calibration
parameters include (for each color):

\begin {itemize}
\item sky - the sky value as determined from the quartiles
\sitem skyslope - the first derivative of the sky in the row direction
\sitem lbias - the difference between the bias level in the bias vector and the
  bias level in the left overscan portion of the image
\sitem rbias - the difference between the bias level in the bias vector and the
  bias level in the right overscan portion of the image
\sitem mag20 - the number of ADU a 20th magnitude star would have accumulated
  in the image
\sitem mag20Err - the error in mag20
\sitem psf - the double Gaussian paramters that best fit the PSF in the image
\sitem toRefcolor - the six linear coefficients that transform row and column
  in each color to row and column in the reference color (probably red)
\sitem toGCC - the six linear coefficients that transform row and column
  in the image into sky coordinates (from astrometric pipeline)
\end {itemize}

\subsubsection{Last calibration}

This gives
the photometric calibration to be used if no photometric calibration
stars are available in the next run.  

\subsection{Data flow in the Frames Pipeline Processor}

	The main program of the processor is a TCL script which accomplishes
the tasks in the accompanying figure.  This main TCL script will control
which tasks are performed on which CPU, and in general can only be restarted 
at a new five color frame.  The diagram shows only the scientific steps
performed.

%\epsfxsize=0.98\textwidth
%\vspace{-1.5cm}
%\epsfbox{ppflow.eps}
\epsfigure{frames_pipeline.eps}{}{}

\subsection{Input to the Frames Pipeline Processor}

	The level zero photometric pipeline reads its input as a
set of FITS and ascii files on disk.  The input includes the data
processing plan, the software parameters, the CCD parameters, the bias
vectors, the calibration parameters, the flatfield vectors, and the
raw images.

\subsubsection{Data Processing Plan}

	The data processing plan identifies the other files to be used in
the processing, a resource plan (tape
devices, disk directories, etc.) as well as the mode of operation (this is
where we can specify debugging, use only two colors, or
normal mode, for example).

\subsubsection{Software Parameters}

The science modules access this file of parameters.  These parameters are
tunable thresholds, radii, etc.  The file is ASCII with one parameter name
and value per line.

\subsubsection{CCD Parameters}

This is currently an ASCII file of parameters, but one could easily
convert it to a FITS file if necessary.  It gives the size and location
of the data regions, the size and location of the bias regions, the
readout noise(s) and gain(s), the defects, etc.

\subsubsection{Calibration Parameters}

The postage stamp pipeline outputs one set of calibration parameters per
frame.  The parameters include: photmetric parameters, astrometric
parameters, psf, sky value and slope, and left and right bias level correction.

\subsubsection{Flatfielding vectors}

From the postage stamp pipeline, we get one bias vector per CCD, and one
flatfield vector for each frame and color.  These are read in as separate
FITS files.

\subsubsection{Raw Images}

The raw images are read in as FITS disk files of frames that are 2128 columns
(including 80 overscan columns) and 1358 rows.

\subsection{Scientific output from the Frames Pipeline Processor}

The output described here should be
sufficient to select galaxy targets for the spectroscopic survey, and to
achieve the primary science outlined in the Principles of Operation of
the Sky Survey Project.  Each of the following is output for each
(five-color) frame.

\subsubsection{Corrected frames}

	The corrected frames are 1489 rows by 2048 columns (including
the 10\% overlaps).  They are currenly written to disk, but one would imagine
that the level one system would have to write them directly to tape.
The frames are bias subtracted and divided by
the flatfield vector (all scaled appropriately).  These files are 6.1
megabytes per color.

\subsubsection{Noise frames}

	The noise frames are one byte of data per pixel in the corrected
frames, giving the error in the pixel value.  This is in general the
Poisson noise, but could be larger due to interpolation, object subtraction,
etc.  The noise is 3.1 megabytes per color.

\subsubsection{Mask}

	The mask is one byte of bit information per pixel in the corrected
frames.  The bits are:

\begin{itemize}
\item MASK\_INTERP - the pixel's value has been interpolated
\item MASK\_SATUR - the pixel is/was saturated
\item MASK\_NOTCHECKED - the pixel was NOT examined for an object
\item MASK\_OBJECT - the pixel is part of some object
\item MASK\_BRIGHTOBJECT - the pixel is part of bright object
\item MASK\_CATOBJECT - the pixel is part of a catalogued object
\item MASK\_SUBTRACTED - the model has been subtracted from pixel
\item MASK\_GHOST - the pixel is part of a ghost
\end {itemize}

The mask is 3.1 megabytes per color.

\subsubsection{Catalog of objects and atlas images}

The catalog of objects is a FITS file containing all of the merged objects.
The list contains all of the objects found by both the bright object and
the faint object finder, and all are classified by the faint object finder
so all of the parameters are comparable.  The FITS file also contains the
atlas images for the objects.  In the nine frames we processed, the
output lists were 119, 3, 10, 12, 13, 13, 11, 31, and 14 megabytes (including
all five colors).  The largest output file was from an image that contained
two crossing satellite trails.

\subsubsection{Binned images}

In addition to the atlas images of the interesting objects in the image,
we store a crude reproduction of the whole image, which will
be useful for quality assurance, and as a tool for finding charts, etc.
We bin the image in 4x4 pixel chunks, and write out the resulting FITS
file.  This output is 0.4 megabytes per color.

\subsubsection{Frame statistics}

	There are several obvious plots we can make to analyse the
results on a frame by frame basis.  To this end, we output a for each frame
the:
\begin{enumerate}
\item number of stars, galaxies, and others (three numbers)
\sitem image minimum, maximum, median, sky, and sigma in each color
\sitem number of bright and faint objects found in each color
\sitem number of pixels flagged as bad in each color.
\end{enumerate}
From this information which is written out as flat files, we can generate
a plot versus frame number for each item.  This is a very small amount of
data.

\subsection{Verbose Output from the Processor}

	If verbose mode is requested, we will get extra information
in addition to the output obtained from normal operation.

\subsection{Images to the screen}

	If display is chosen, raw frames, and corrected frames, 
noise regions, and (overlaid) 
masks are sent to a display, and the objects as they are found are indicated
on the corrected image.
