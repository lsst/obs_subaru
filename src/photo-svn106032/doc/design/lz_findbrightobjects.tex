\def \bom{{\bf Find Bright Objects\ }}

\subsection{Introduction}

The find bright objects module is responsible for finding all objects
whose peaks are above some theshold.  This will probably be done as
some connected pixel map(s).  For one map, the algorithm would be:

\begin{verbatim}
     Scan array for pixels 100 sigma above the sky noise
                        |
        Tag these pixels in a temporary mask
                        |
     For each tagged pixel find all connected pixels
     down to 0.5 sigma of the sky and write to a structure
                        |
  Identify saturated pixels and calculate the moments Wxx Wxy Wyy
               of the connected pixels
                        |
  For star-like objects, fit the model PSF to the object to
    determine whether it's really a star, and also to
    measure its position and brightness most accurately
\end{verbatim}

where the values ``100'' and ``0.5'' are examples of the
parameters given as input to this module (see {\bf Input} section).

\subsection {InitModule}

\subsubsection {Description}

  The initialization routine sets parameters that the main module will
use in identifying objects as ``Bright''.  It may also read information
used by some of the sub-modules which specialize in specific types of
``contaminants.''

\subsubsection {Input}

  The {\bf InitModule} reads parameters from the ``parameters file''
into memory which is private to the \bom module.  I suggest that it
use these values in particular:

\begin{itemize}
  \item a threshold value above which pixels are classified as being
        ``Bright;''  this may be either an absolute Data Number,
        or some multiple of the sky sigma.  Any object which contains
        a pixel above this threshold is a ``Bright Object.''
  \item another threshold, down to which it will extend objects from their
        bright portions;  this should be some small multiple of the sky
        sigma
  \item (optionally) a value above which pixels are regarded as BAD
        because the Poisson noise in the number of incoming photons is
        so much larger than the typical pixel noise values that it
        becomes impossible to model and subtract them to yield useful
        ``underlying'' data
\end{itemize}

  However, after some experimentation, we may find that some other
kind of parameters are more appropriate; these are simply suggestions.

\subsubsection {Output}

  There is no output under normal operation.

\subsubsection {Modules}

  The only job of the {\bf InitModule} is to read parameter values
into memory.

\subsubsection {Algorithms}

  Any simple algorithm will do.

\subsubsection {Quality, Debugging, Resources}

  The {\bf InitModule} should print error messages and call the standard
error-handling routine if it is unable to read the appropriate parameter
values.

  In ``debug mode,'' the function should print out the value of
each parameter as it succeeds in reading it from the parameters file.

  This routine will consume negligable resources.

\subsubsection {Test Data Required}

  There is no test data required.

\subsubsection {Regression Testing}

  The ``debug mode'' output should suffice for regression testing.

\subsection{Input}

\subsubsection{Object finding parameters}
\subsubsection{A 5-color frame}

\subsection{Output}

\subsubsection{Connected pixel map(s)}
\subsubsection{Object list}
This will contain the position of the center of each connected region
in the highest mask level in which it appears.

\subsection{Modules}

\subsection{Algorithms}

\subsection{Quality, Debugging, Resources}


  Quality Assurance is a very difficult task for this
module, because the number, relative frequency and
severity of bright stuff vary greatly from one
frame to the next.  One check would be to examine the
list of Bright Objects created by the module to see
if all the known, catalogued strange objects were
found.  It would be useful for the module to print
a summary, in several lines, of all the things it
found for a given frame (i.e., ``15 cosmic rays,
3 bleed trails, 1 ghost'').

  For debugging and QA, it should be possible to
display an image and blink on/off all those pixels
belonging to identified Bright Objects, all those
marked as BAD, etc.

\subsection {FiniModule}

\subsubsection {Description}

  The {\bf FiniModule} basically has to free any memory
which was allocated by the {\bf InitModule}.  However,
it would be useful if it also could, upon demand,
produce a list (either of structures, or an ASCII file)
summarizing the results of \bom on all frames processed
in the current run.

\subsubsection {Input}

  The only input comes from information the module
has been keeping in private data space during the
module operation.

\subsubsection {Output}

  No output is required from the code which frees memory.

  I suggest that the {\bf FiniModule} produce an ASCII
list summarizing the number and type of Bright Objects
detected during the entire pipeline execution.
This list could be used for QA, debugging, or as a
scientific product in its own right.

\subsubsection {Modules}

  There are two functions:

\begin{itemize}
  \item free memory
  \item create and print summary
\end{itemize}

\subsubsection {Algorithms}

  Both functions can use straightforward methods.
We need only simple statistics in the summary.

\subsubsection {Quality, Debugging, Resources}

  The memory-freeing routine will call standard error
functions if it has a problem.  It will consume negligible
resources.

  The summary routine might make sanity checks
on its statistics before printing them (for example,
it might check to see if there were negative numbers
of any sort of defect, or more defects than pixels, etc.).
It would help pipeline operators if they knew what
kind of summary was typical; thus, the following
might be nice: ``there were 25 cosmic rays per frame on
average ... (typical value: 23).''

  No significant amounts of memory or CPU time will
be required.

