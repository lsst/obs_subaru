\def\bigskip{\vspace{20pt}}
\def\smallstep{vspace{10pt}}
\def\ltsima{$\; \buildrel < \over \sim \;$}
\def\gtsima{$\; \buildrel > \over \sim \;$}

\bigskip
\begin{center}
\section{OVERVIEW}
\smallskip
\end{center}

This section provides a guide for users of the SDSS simulated test data.
A catalog contains galaxies (of various types) and stars.
The numbers in a catalog file are supposed to represent intrinsic
parameters of these objects, the sort of quantities that we might imagine
storing in the final archive.  The image code turns a catalog into a
set of images (in different bands), representing each
object by a combination of exponential disk, de Vaucouleurs-law bulge, 
and point source.  Atmospheric and instrumental effects --- 
seeing, extinction, noise, cosmic rays, etc.\ --- are added at this
stage; they do not influence the catalog itself.

\bigskip
\begin{center}
\section{CATALOGS}
\end{center}
\smallskip

A catalog is a binary data file with one header structure that describes
global properties of the catalog followed by $N$ object structures,
where $N$ is the total number of objects in the catalog.
The catalog header is a structure of the following form:
\begin{verbatim}
struct dsscath_ty{       /* catalog header quantities */
    int   hd_nobj;      /* number of objects */
    int   hd_rev_g;     /* software revision, galaxy code */
    int   hd_rev_s;     /* software revision, star code */
    int   hd_rev_q;     /* software revision, quasar code */
    int   hd_mdver;     /* morphology-density relation version */
    int   hd_redver;    /* reddening map version */
    int   hd_seed_g;    /* random number seed used for galaxy simulation */
    int   hd_seed_s;    /* random number seed used for star simulation */
    int   hd_seed_q;    /* random number seed used for quasar simulation */
    int   hd_xsize;
    int   hd_ysize;     /* pic x and y sizes, in pixels */
    float hd_scale;     /* scale in arcsec/pixel */
    float hd_phi;       /* phi of field in galaxy simulation, degrees */
    float hd_theta;     /* theta of field in galaxy simulation, degrees */
    float hd_gal_l;     /* galactic longitude of field used in stars 
                           simulation, decimal degrees */
    float hd_gal_b;     /* galactic latitude of field used in stars 
                           simulation, decimal degrees */
    float hd_maglim[NFILTER];   /* catalog magnitude limits */
    int   hd_ispare[64];
    float hd_fspare[64];
}
\end{verbatim}

Different codes are used to add galaxies, stars, and quasars to the catalog;
revision numbers and random seeds are listed for each of these codes.
(NB: the quasar code does not exist yet.)  There are also entries for
the version of the morphology-density relation used to assign galaxy
types and for the galactic reddening map used to create the catalog.  
Because reddening happens above the atmosphere, it goes into the catalog,
not into the transition from catalog to images.
(NB: reddening
is not implemented yet.)  The {\tt hd\_theta} and {\tt hd\_phi} entries
specify the line-of-sight used to select galaxies from the N-body simulation;
the {\tt hd\_gal\_l} and {\tt hd\_gal\_b} entries specify the galactic
coordinates assumed for adding stars.  These quantities are presently
independent, but in the future we will probably adopt a coordinate system
that ties both to survey coordinates.  There are magnitude limits for
each filter; an object makes it into the catalog if it lies above the
magnitude limit in at least one filter.

For each object, there is a data structure of the following form:
\begin{verbatim}
struct dsssim_ty{	    	    /* catalog object quantities */
    float s_x;                  /* position in pixels */
    float s_y;
    float s_tflux;              /* total r-band flux density, in Jy */
    float s_z;                  /* redshift */
    float s_tanthet;            /* tangent of P. A. */
    float s_pfrac;              /* fraction of light in psf */
    float s_bfrac;              /* fraction of light in DeV bulge */
    float s_color_p[NFILTER];   /* colors (ratio to r-band flux density) 
                                                         of psf component*/
    float s_color_b[NFILTER];   /* ditto for bulge component */
    float s_color_d[NFILTER];   /* ditto for exponential disk component */
    float s_axrat_b;            /* proj. bulge axis ratio (btwn 0 and 1) */
    float s_axrat_d;            /* ditto for exp disk */
    float s_reff_b;             /* half-light major axis of bulge in pix */
    float s_reff_d;             /* ditto for disk (1.678 x scale length ) */
    int   s_class;              /* int class (0=star 1=quasar 2=E 3=S0 4=Sa,
                                 * 5=Sb 6=Sc 7=Sd/Irr)
                                 */    
    /* following 4 variables for use with scaled real images */
    int   s_catindex ;          /* index in real image catalog */
    float s_multiplier ;        /* flux multiplier */
    float s_scale ;             /* length rescaling */
    char  s_objname[7] ;        /* name of galaxy used for image */
    float s_fspare;             /* float spare */
    int   s_ispare;             /* int spare */
}
\end{verbatim}

An object may have a point source component, a de Vaucouleurs-law 
bulge component, and an exponential disk component.  
The fraction of the $r$-band light in the point source and bulge components
is specified by the quantities {\tt s\_pfrac} and {\tt s\_bfrac}, 
with any remaining light assigned to the disk component.  Each component
may have a different color.  At present, stars and quasars are represented
by point sources, elliptical galaxies by pure bulges, and later type 
galaxies (including S0's)
by pure disks.  However, we plan to model intermediate Hubble
types with combinations of bulge and disk, and perhaps even include
`Seyfert' galaxies with a point source component at the nucleus.
The last four variables will be used with versions of the simulation code
that use scaled images of real galaxies to represent the brighter 
simulated galaxies.  These variables will allow the user to 
determine which real object corresponds to a simulated object, and to 
find out how the image of that object was scaled.

Two C programs that deal with catalog files will be provided along with
the simulated catalogs themselves.  The first, called `cat2ascii',
translates a binary catalog file into ascii format.  Users who want to
write their own code to compare the output of their pipeline modules
to the catalog file may find `cat2ascii' a useful template from which
to start.  The second program, called `ascii2cat', translates an ascii
file (of the sort produced by `cat2ascii') into binary catalog format.
One can create a `template' ascii file by running `cat2ascii' on an
existing catalog, then edit the ascii file with one's favorite editor
and run `ascii2cat' to create a new catalog.  This allows a user to
change the properties of an object in a catalog, to delete objects from
a catalog, to add objects to a catalog, or to create a specialized catalog
with complex or extreme objects  of particular interest, 
e.g.\ a star in the middle of a binary galaxy.  Note that {\tt nobj} in
the header must be changed to correspond to the number of objects in
the revised catalog.

\begin{center}
\bigskip
\section{IMAGES}
\smallskip
\end{center}

This section describes the methods used to create simulated images
corresponding to a given catalog.  In the current version of the image-making
code, a catalog object is represented by a sum of up to three profiles:
a point spread function (PSF), an inclined exponential disk convolved with
a PSF, and a de Vaucoleurs-law bulge convolved with a PSF.
We first describe these profiles, then summarize our procedures for
adding noise and defects to the images and list some of the parameters
that are used to define the images.
  
\vspace{10pt}
\subsection{ PSF }

For the form of the point spread function we adopt a
double-Gaussian with an $r^{-3}$ power-law tail.
Thus, the intensity pattern produced by a point source of total flux $F_0$ is
\begin{equation}
I(r) = {F_0 \over K}
       \left[ \alpha e^{-{r^2}/2{\sigma_1}^2}
       + (1 - \alpha - f) e^{-{r^2}/2{\sigma_2}^2}
       + f\left(1 + \frac{r^2}{\beta {{\sigma}_2}^2}\right)^{-\beta/2}\right] ~,
\end{equation}
with 
$\alpha = 0.9, \beta = 3, \sigma_1 = 0.94$ (pixel), 
$\sigma_2$ = 2.7 (pixel), and
\begin{equation}
f = \frac{\omega (\beta - 2)(\alpha \frac{{\sigma_1}^2}{{\sigma_2}^2} +
    (1 - \alpha))}{\beta - 2 \omega}
\end{equation}
where $\omega$ = 0.075 is the fraction of the total
light in the power-law component.
The normalization constant is
\begin{equation}
K = 2\pi\left[\alpha\sigma_1^2 + (1-\alpha-f)\sigma_2^2 +
        f\beta\sigma_2^2/(\beta-2)\right] ~.
\end{equation}
With these parameters,
the PSF corresponds approximately to 1 arcsec seeing.

We add diffraction spikes to all the stars.
The intensity pattern produced by a diffraction spike is:
\begin{equation}
I_{\rm spider}(r,j) = spk_{\rm amp}(r) * spk_{\rm cross}(j) ~,
\end{equation}
where
\begin{equation}
spk_{\rm amp}(r) = \frac{I_0 \gamma \delta }{\gamma^2 + r^2} ~,
\end{equation}
$\gamma$ = 10 (pixel), $\delta$ = 0.01, $I_0$ is the central intensity 
of the star, and
\begin{equation}
spk_{\rm cross}(j) = \frac{1}{\sqrt{2 \pi}}
      \left[\alpha \sigma_1 e^{-j^2/2{\sigma_1}^2}
      + (1 - \alpha) \sigma_2 e^{-j^2/2{\sigma_2}^2}\right] ~.
\end{equation}
Here $r$ represents a distance from the central star along an arm of the
diffraction cross, and $j$ represents a distance transverse to the arm.
Thus $spk_{\rm amp}(r)$ represents the ``raw'' intensity pattern of the
diffraction spike, and the convolution with $spk_{\rm cross}(j)$
incorporates the effect of seeing (note that we do not include the
power-law tail of the seeing in the diffraction spikes).
 
Saturation tracks are made in the central peak of very bright stars, along 
the direction of the charge transfer (the scanning direction = $y$-axis).
We set the full-well saturation level at 300,000 electrons.
(In u and g band, the cutoff produced by 16-bit truncation of the 
analog-to-digital converter takes effect before CCD full-well saturation).

\vspace{10pt}
\subsection{ Galaxy Profiles }

For galaxies, we use two kinds of profile, exponential law and de Vaucouleurs
law.  

The exponential profiles of observed disk galaxies tend to cut off sharply
at about four scale lengths.  Therefore, we represent disks by a 
truncated exponential:
\begin{equation}
I(r) = I_0 (e^{-{r}/{\epsilon r_e}} - e^{-{4}/{\epsilon r_e}})  
\end{equation}
for ($r \leq 4 \epsilon r_e$), and
\begin{equation}
I(r) = 0  
\end{equation}
for ($r > 4 \epsilon r_e$).
Here $\epsilon = 1.6783^{-1}$ is the conversion factor between the 
exponential scale length and the effective radius $r_e$ containing half
the total light: $r_s=\epsilon r_e$.

The spheroidal component is represented by a de Vaucouleurs law,
\begin{equation}
I(r) = I_0 e^{-({r}/{\zeta r_e})^{1/4}} ~,
\end{equation}
where $\zeta = 7.6692^{-1}$ is the conversion factor between the scale
length of the de Vaucoleurs law and the effective radius $r_e$.
At present, these profiles are truncated when $r$ exceeds 192 pixels,
or at smaller radii if the intensity becomes insignificant.

The profiles described above are convolved with the PSF of equation (1).
For reasons of computational efficiency, we calculate the intensity
pattern of a given object by interpolation within a grid of 276 template
patterns, which represent profiles with different axis ratios and sizes.
For exponential disks, we use seven template axis ratios
(1.0, 0.8, 0.6, 0.4, 0.25, 0.16, 0.10) and compute the inclined profiles
assuming that the disks are optically thin.
For de Vaucoleurs-law spheroids, we use five template axis ratios
(1.0, 0.8, 0.6, 0.4, 0.28).  We make 23 templates for each axis ratio,
with effective radius increasing by factors of $10^{0.1}$ from
0.25 (0.4) to 39.6 (15.9) pixels for exponential (de Vaucoleurs-law) profiles.
In the unusual circumstance that a galaxy lies outside this range of radius,
we extrapolate.

\vspace{10pt}
\subsection{Dithering}

When the signal produced by a profile in a single pixel falls below 1 DN,
we extend the profile further by dithering. 
When the value is 0.7, for example,
we input the value 1 DN with a probability of 70\% and 0 DN 
with probability of 30\%.  Objects thus have dithered tails even when 
they are in noiseless images.   

\vspace{10pt}
\subsection{Noise and Defects}

For each catalog and filter, we make images with varying degrees of noise
and other defects:

\begin{quote}
    (1)  noiseless images

    (2) images with readout noise and photon noise

    (3) images with photon noise and a background taken directly from 
	a test CCD (including cosmic rays)

    (4) everything in (3), plus satellite tracks.

    (5) everything in (4), plus sky background fluctuation,
        band-dependent positional shifts of objects due to atmospheric 
        differential refraction and rotation of the diffraction spikes.

    (6) everything in (5), plus real CCD background, four "real"
        galaxy images, and GSC stars.
\end{quote}

In (2), we assume that the sum of the readout and photon noise is Gaussian.
In (3) we take the readout noise directly from one of the test CCD's.
Most of (4) is not yet implemented, but we plan to 
add these effects and others in the future.  

\vspace{10pt}
\subsection{Un-flattened Images}

The images described so far are flat and unbiased.  This is fine for
testing most of the pipeline modules, which operate after {\tt CorrectFrames}
has done its thing.  However, for some tests we also need un-flattened
images, with intensity
\begin{equation}
I_{unflat} = \frac{I_{flat}}{FF} + bias ~.
\end{equation}
Here $FF$ and $bias$ are flat-field and bias vectors, which are obtained
directly from test CCD data and provided together with the simulated images.
\footnote{We are using a set of flat-field vectors and bias vectors provided by
Marc Postman.  Please ask him if you have any questions about these vectors.}
To separate Postage Stamp Pipeline from Frames Pipeline, we also
provided flattened images with unflatened format.
The unflatenned images are stored in frames\_6 of photodata input, and
flatenned images with unflatened format are stored in frames\_(please ask Naoki).

\vspace{10pt}
\subsection{Parameters}
The image scale is 0.400 arcsec/pixel.
Readout noise is 5 electrons rms.
The AD unit is determined so that 1 DN corresponds to 2, 4, 6, 6, and
6 electrons from u to z band.

All magnitudes are on the AB system (where 0 mag represents a
flux-density of 3630 Jy).
A 1 Jy flux density gives $qfactor \times qtdll$ electrons, where 
$qfactor$ = 3.02e9 and $qtdll$ = 0.047, 0.121, 0.120, 0.082, and 0.021
for u, g, r, i, and z band respectively.  This value of $qfactor$
assumes an exposure time of 55 seconds and secondary obscuration of 26\%.

The assumed sky brightness is 23.25, 22.31, 21.32, 20.46, and 18.97 mag 
${\rm arcsec}^{-2}$ for u', g', r', i', and z' band (These sky surface
brightness levels are recently re-calculated by T.Ichikawa et al. with
higher wavelength resolution). 
Atmospheric extinction is 0.675, 0.223, 0.111, 0.097, and 0.135 from u to z 
at 1.0 airmass.
Please note that sky brightness is defined at the ground.
So you should not use it to convert DNs to magnitudes of stars and galaxies. 

We assume atmospheric differential refraction as 2.0, 0.93, 0.39, 0.16, 0.0 
(arcsec) for u',g',r',i', and z' band for noise level $\geq$5. 
(Eventually we will change these values as a function of secz).

For levelzero test images, we prepared 2x10x5 frames.
We have separated this original catalog
into 2x10 catalogs using the code written by Kazu Shimasaku.  
Filename conventios are described later of this section.

We input sky background variation as a function of time for noise level $\geq$5.
We have neglected spatial variation for the moment.
The functional form is made by a combination of a few sinusoidal
functions with realistic amplitude and period based on the
real data taken at Mt. Hopkins (by S.Kent) and at Palomar
(by M.Postman). The postscript file "skyvar.ps" shows
the functional form within the current strip.  We use the same function
for all five color bands, though we take into account the time delay of
different color bands due to the camera geometry
( r',z',u',i',g' toward the scanning direction).

The background frames are added for noise level $\geq$ 6.
The images are provided by Tim McKay from FNAL DSC.
CCD420.fits - CCD429.fits corresponding to 0th to 9th exposure number
(explained later).
Though each FITS file has 2128x2048 pixels, we use only first
2128x1354 pixels. The same background was used in all
five colors for the convenience.

Four "real" galaxies are included for noise level $\geq$ 6.
The original images made by Heidi are stored in the same directory
as in the simulated images.

\begin{verbatim}
    m51_conv.fit      :   high_1_0_0,
    m51_med_conv.fit  :   high_1_1_0,
    m51_ang_conv.fit  :   high_1_2_0,
    m51_flat_conv.fit :   high_1_3_0.
\end{verbatim}

We multiply 0.1, 0.2, 0.3, 0.4, and 0.5 to the original
values for u',g',r',i', and z' band.
The position of the image (center of the original image) is
(512.5,439.5) [(512.5, 512.5) for flattened images] for z-band
images. Note that these four galaxies are not included in the table.

Real star positions (GSC stars) are included in the images with
noise level $\geq$ 6. 
Since the original catalog doesn't include the color of the stars,
we assumed that the magnitude in the catalog is in r' band,
and multiplied the linear factors 0.1, 0.5, 1.0, 1.5, and 2.0
for u', g', r', i', and z' band respectively. (This is not the 
color but linear factor. If you want to get color, take $-2.5 log_{10}$
of these factors.) 
HOWEVER it turns out that there is a
serious bug in the astrometric conversion of GSC stars, so please do not
use the positions of GSC stars for the moment.

All of these parameters are listed in the parameter file ({\tt *.par}) that
accompanies a simulated image.

\begin{center}
\bigskip
\section{ FILENAME CONVENTIONS AND FILE FORMATS}
\smallskip
\end{center}

\vspace{10pt}
\subsection{Filename Conventions}

Suppose that we have a simulation of a high-galactic latitude field
named {\tt high}.  (We do in fact have such a 
simulation, and a low-galactic latitude simulation called {\tt low}.
In later versions of the simulations, the filenames will indicate the
position of the field in survey coordinates.)  
The complete set of simulated data for one field will include the following 
files:
\begin{verbatim}
    high_0_9_1.cat, high_0_9_1.log,
    high_0_9_1_u6_unflat.fit, 
    high_0_9_1_u6_unflat.par, high_0_9_1_u6_unflat.lst,
    high_0_9_1_g6_unflat.fit,
    high_0_9_1_g6_unflat.par, high_0_9_1_g6_unflat.lst,
    high_0_9_1_r6_unflat.fit,
    high_0_9_1_r6_unflat.par, high_0_9_1_r6_unflat.lst,
    high_0_9_1_i6_unflat.fit,
    high_0_9_1_i6_unflat.par, high_0_9_1_i6_unflat.lst,
    high_0_9_1_z6_unflat.fit,
    high_0_9_1_z6_unflat.par, high_0_9_1_z6_unflat.lst,
\end{verbatim}

Suppose taht the file {\tt high.cat} is the original catalog file, 
in the binary format described above.
For level zero test images, we have separated this original catalog 
into 2x10 catalogs high\_(a)\_(b)\_(c).cat.
Here (a) indicates strip No. (0 and 1 makes one stripe), 
(b) indicates the number of the exposure, and (c) indicates the 
position of the dewar (or column of the CCD). 
There are simple observational log files which include
the survey coordinate (eta,lambda), Date, sec(z), Galactic coordinate (l,b)
high\_(a)\_(b)\_(c).log.
The letter following {\tt \_(a)\_(b)\_(c)} indicates the filter, i.e.\ u, g,
r, i, z (omitting the $^\prime$s).
The number $6$ following the filter
name indicates the noise/defect category of the image.
Unflattened images have {\tt \_unflat} after the filter/noise
entry, e.g.\ {\tt high\_0\_9\_1\_u1\_unflat.fit}, and we provide 1-d fits
files containing the flat-field and bias vectors with names 
{\tt high\_u\_ff.fit}, {\tt high\_bias.fit}, etc. 

Files ending in {\tt .fit} are the images, in FITS U16 format.
The image size of unflattened images is 2128 $\times$
1354 with 20 extended and 20 overscanning pixels in each side.
Files ending in {\tt .par} are ASCII files listing
parameters used to create the images.  Files ending in {\tt .lst} 
are ASCII files listing properties of the objects in the image
(and are therefore partly redundant with the catalog file).


\vspace{10pt}
\subsection{File Formats}

The format of the catalog file has already been described.
The format of the remaining files is as follows.

\noindent
i) {\tt *.fit} (FITS file)

The image files are in FITS U16 format, with size 2048 $\times$ 1354 pixels.
The fits header of the test images is based on the document
``TEST DATA PREPROCESSOR SPECIFICATIONS''
by S. Kent et al.
Some of the header entries carry ``real'' meaning,
but most of them are still dummy parameters. We will assign realistic
values to these parameters when we create a succession of
test frames along a stripe.

\noindent
ii) {\tt *.par} (ASCII file)

\noindent
This file lists the parameters adopted when creating the image, e.g.\

\begin{quote}
Double gaussian psf, sig1,2 = 0.94 2.70, amp1=0.900

e- per Jy = 3.62e+08, e- per DN =  6.00

filter r'  lameff  6279  sig 0.071

noise : photon

atmospheric extinction (mag) = 0.130647

sky background (mag)         = 21.320000

background level (DN)        = 104

saturation level  = 50000

software revision = 7

\end{quote}
The last entry represents the revision number of the image-making code,
not the code used to make the catalog itself.  Catalog software versions
are listed in the catalog header.
Since we already experimented with a few test images using an older version of 
the image code, we list the software revision as 2 for these first
archive images.

When we make an image, we remove extremely bright or faint objects.
The range of magnitudes used for different object classes is listed in 
the {\tt .par} file like this:

\begin{tabular}{lrrrrrrrr} 
flux range (class)  &   0 &    1 &    2 &    3 &    4 &    5 &    6  &   7 \\

 max (mag)        = &  0.0&   0.0&   0.0&   0.0&   0.0&   0.0&   0.0 &  0.0 \\

 min (mag)        = & 30.0&  30.0&  30.0&  30.0&  30.0&  30.0&  30.0 & 30.0 \\
\end{tabular} 
\begin{quote} 
number of objects used =   1075 /   1075  
\end{quote} 
The last line tells what fraction of the objects in the catalog file
actually went into the image.  Note that this number may vary from band 
to band, since an object may be within the flux limits in one band
but outside them in another.

The {\tt .par} file ends with a table that
shows the number of objects contained in the frame 
by 1-mag bin and by class.

\begin{tabular} {rlrcrrrrrrrrr}
 class & mag &$<$  10.0& ...&  16.0&  17.0&  18.0&  19.0&  20.0&  21.0  &22.0&  23.0&  24.0$<$ \\

    0& star   & 0 &...&  9 &   7  & 12  & 13  & 31  & 34 &  31  & 53  &171\\

    1& quasar & 0 &...&  0 &   0  &  0  &  0  &  0  &  0 &   0  &  0  &  0\\

    2& E      & 0 &...&  0 &   0  &  1  &  1  &  7  &  8 &  27  & 34  & 64\\

    3& S0     & 0 &...&  0 &   0  &  1  &  2  &  9  & 26 &  55  & 76  &131\\

    4& Sa     & 0 &...&  0 &   1  &  0  &  4  &  6  & 20 &  52  & 76  &118\\

    5& Sb     & 0 &...&  0 &   1  &  4  &  7  &  9  & 35 &  68  & 83  &113\\

    6& Sc     & 0 &...&  0 &   1  &  2  &  8  & 17  & 53 &  96  &123  &155\\

    7& Sd/Irr & 0 &...&  0 &   0  &  0  &  0  &  0  &  0 &   0  &  0  &  0\\

\end{tabular} 

\noindent
iii) {\tt *.lst} (ASCII file)

Because of the limitations of the image creation process, we expect 
differences of order one percent
between the flux in the catalog and the flux in the image itself.
We also make approximations and interpolations for the shapes of the galaxies.
This file lists properties of the objects as they appear in the image.

The first 17 lines of this file are exactly the same as those in 
an ASCII translation of the catalog file {\tt *.cat}.
We add one line which shows the color and the units used in the image,
angle toward the zenith and sec z for the user's convenience,
e.g., 

\begin{quote}
filter = r'  1 Jy =  3.213e+08 electrons, 1 Jy =  5.355e+07 DN, 1 DN =    28.22 mag           

angle(deg)=-0.551173   sec z=1.177000

\end{quote}

\noindent
Then there is a table of objects in the frame.
One line corresponds to one object. 

\begin{tabular}{ll}
{\bf No}  & sequential number of the object in the image \\

{\bf r}   &    position (pixel) in row  \\

{\bf c}   &    position (pixel) in column  \\

{\bf f\_c(DN)} &flux (DN) in the catalog \\

{\bf r\_b} &  effective radius (pixel) for deV component in the catalog \\

{\bf r\_d} &   effective radius (pixel) for exp component in the catalog \\

{\bf ar\_b}&   axis ratio for deV component in the catalog \\

{\bf ar\_d}&   axis ratio for exp component in the catalog  \\

{\bf tanth}&   tangent of position angle in the catalog  \\

{\bf f\_in(DN)}& flux (DN) input in the image \\

{\bf aj}  &   template index in the image  \\

{\bf f\_cat(Jy)}& flux (Jy) in the catalog \\

{\bf color\_p} & flux ratio of PSF component in this band with respect to r-band flux \\

{\bf color\_b} &  flux ratio of deV component in this band with respect to r-band flux \\

{\bf color\_d} &  flux ratio of exp component in this band with respect to r-band flux \\

{\bf cl}     &  class (0=star 1=quasar 2=E 3=S0 4=Sa, 5=Sb 6=Sc 7=Sd/Irr) \\

{\bf redshift}     &  redshift of the object \\
\end{tabular}

Most of this information is redundant with that in the catalog file.
The important new piece of information is the flux that actually appears
in the image.  The index $aj$ indicates, in a rather obscure way,
which template was used to create the object image.  We don't expect
that end-users will need this information; those who do should contact
MD or JG.

\begin{center}
\bigskip
\section{ Other  Simulated Images}
\smallskip
\end{center}

We are mainly providing noise level 6 (according to module writers'
request). M.D. personally feels that we should use noise level 2 
as a first step. M.D. is glad to provide special purpose images such as 
with different noise levels, the images without diffraction spikes, the 
images without galaxies, etc. The images for the Monitor telescopes are 
also created with the observational parameters for MT.
If you need special purpose test data, please contact with M.D. or D.H.W.


