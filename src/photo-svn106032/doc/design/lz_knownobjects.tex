
\def \kom{{\bf Known Objects\ }}
\def \catobj{CAT\_OBJ }

\subsection {Introduction}

  The \kom module runs once per strip.
It tries to find any objects that are listed
in a set of catalogs which might fall entirely
or partially within any frame in the current strip.
It creates a linked
list of all such objects, which is be passed by the 
framework to other modules
in the pipeline as necessary.

  Note that the task of compiling a set of catalogs of known 
objects across the entire Survey Area, reducing them all
to a common format, and sorting them all into a standard
order is a formidable one. 
However, it may certainly be done in an incremental 
fashion --- there is no need for the Level Zero catalogs to
be completely comprehensive in every way.
Keep in mind that this module can do a job only as good as
the information in the catalogs; the {\it Garbage In, Garbage
Out} precept applies strongly in this module.

\subsection{InitModule}

This opens all the necessary files.

\subsection {Module}

\subsubsection {Description}

  The \kom module searches through a set of catalogs, looking
for objects that might fall into the current strip.  It creates
a structure containing catalog information for each such object
and links all the structures together in a list.

\subsubsection {Input}

  This module requires

\begin{itemize}
  \item the RA, Dec, equinox and time (date + UT) of start of the current 
        strip
  \item the RA, Dec, equinox and time (date + UT) of end of the current 
        strip
  \item the width of the strip
  \item the ratio of arcseconds to pixels
  \item a set of file pointers, one per catalog (or the equivalent
        framework structure(s))
\end{itemize}

  The information in each catalog {\it must} be arranged
in the same fashion, which will allow quick access to those
objects which lie within a small portion of the sky.
I suggest that each catalog be sorted primarily by RA,
and secondarily by Dec.  Further, it would increase
this module's efficiency if the information in 
each catalog was in the same form, and if that form 
contained fields similar to the fields of the 
{\it \catobj} structure.

  Steve Kent has recommended a format for the catalogs that
contains a FITS (or FITS-like) header, followed by entries
for each object.  A full description of the format is
given in the Astrometric Pipeline document.  

  Of course, we could combine all the catalog information 
into a single file and simplify the pipeline code, but I
suspect that it will require extra effort to compile and
update.

  Note that we will have to spend extra effort making 
a catalog of asteroid and planet positions, since they change
rapidly with time.  The format of the catalog might have to be
different, and/or we'd need a special function to update
all positions to the time of exposure.

\subsubsection {Output}

  The module produces a linked list of {\it \catobj} structures,
one for each object in the strip.  If no objects are found,
the module produces a NULL pointer.

\subsubsection {Modules}

  The main part of the \kom module will perform the following
tasks, for each of the catalog file pointers in turn:

\begin{itemize}
  \item read items from the catalog sequentially (which may
        take place through the pipeline framework); for each one
  \begin{itemize}
     \item convert item's position to equinox of frame
           (and calculate position at time of observation, for
           asteroids, planets, etc.)
     \item check to see if item falls at least partially within
           strip.  If not, skip to next item; otherwise
     \item create a new {\it \catobj} structure and fill it with
           the appropriate information
     \item add the new structure to the linked list
  \end{itemize}
\end{itemize}

After all catalogs have been read and the list is complete,
the module must sort the list by $\lambda$-coordinate
(the coordinate along the great circle of the strip which
increases from start to end of the strip)
to speed up access for future users of the list.

After finishing all reads, the module should tell the framework
to close all the file pointers.  

\subsubsection {Algorithms}

  We will need functions to precess the coordinates
of objects to the equinox of exposure, and also functions
to calculate the positions of bodies in the solar system.
Fortunately, there are many sources for such routines
in the literature (for example, Jan Meeus' {\it Astronomical
Algorithms}).

\subsubsection {Quality, Debugging, Resources}

  We can ensure that the \kom module is working correctly by
making a list of all the known objects that ought to be selected
for several different specific image coordinates.
Running the module with those coordinates, we can check to make
sure that the list of objects found by the module matches the
correct, pre-arranged list.

  During the debugging phase, it may be useful to print out
a line for each catalog searched, giving the number of objects
listed in the catalog and the number which fall within the
field of view.  In the most verbose ``debug mode,'' the code
might print out a line per item in each catalog, giving its
name and distance from the center of the field or from the 
closest edge of the field.

  If the catalogs contain many thousands or tens of thousands
of objects, we may have to work hard to optimize catalog look-up.
This module should take only a negligible fraction of the total
pipeline memory and CPU time in the Level Zero implementation
(and all subsequent ones).

  The amount of memory required to store the output linked list
of objects in the strip will be roughly eighty bytes per object
(the number varies with the length and number of object names)
times the number of objects in the strip.  Assuming that the 
catalogs we choose as input contain one million objects over the 
entire sky, a single strip will contain around ten thousand 
objects; thus, each strip's list will contain less than one Megabyte
of memory.

\subsubsection {Test Data Required}

  We must create accurate lists of known objects that fall within
several different fields of view, so that we can check the performance
of the \kom module against them.  Once a set of catalogs has been
created, this additional task will require a minimal amount of time.

\subsubsection {Regression Testing}

  The same accurate lists of known objects needed for Test Data
(see preceeding section) can be used for Regression Testing. 
Note that, whenever a change is made to one or more of the
catalogs, the accurate lists must be remade and checked.
A set of five to ten fields with accurate lists should be
sufficient for testing purposes.

\subsection {FiniModule}

\subsubsection {Description}

  The {\bf FiniModule} routine de-allocates the memory used by
the main module for the linked list of objects in the strip.

\subsubsection {Input}

  A pointer to the list of {\it CAT\_OBJ} structures.

\subsubsection {Output}

  There is no output.

\subsubsection {Modules}

  A single function can deallocate the memory.

\subsubsection {Algorithms}

  No fancy algorithms are required.

\subsubsection {Quality, Debugging, Resources}

  Nothing required, unless we want to check each {\it CAT\_OBJ}
structure before deleting it.

\subsubsection {Test Data Required}

  No test data is required.

\subsubsection {Regression Testing}

  None required.

% end of section


